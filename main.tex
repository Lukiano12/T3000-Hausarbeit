\documentclass[a4paper, 12pt]{article} % Artikel-Klasse

%---------------------------------------------------------
% Encoding, language, quotes
%---------------------------------------------------------
\usepackage[utf8]{inputenc}
\usepackage[ngerman]{babel}       % Deutsche Sprache und Silbentrennung
\usepackage{csquotes}             % Für korrekte Anführungszeichen
\usepackage{listings}
\usepackage{xcolor}

%---------------------------------------------------------
% Graphics & PDF
%---------------------------------------------------------
\usepackage{graphicx}
\usepackage{pdfpages}             % Einbinden von PDF-Seiten
\usepackage{caption}              % Verbesserte Bildunterschriften
\usepackage{subcaption}
\usepackage{comment}


% Customize settings
% Customize settings for a more compact style
\lstset{
    language=Java,                 % Specify Java language for syntax highlighting
    basicstyle=\ttfamily\small,    % Use smaller monospaced font for code
    keywordstyle=\color{blue},     % Style for keywords
    commentstyle=\color{gray},     % Style for comments
    stringstyle=\color{red},       % Style for strings
    numbers=none,                  % No line numbers
    breaklines=true,               % Break long lines
    frame=none,                    % No frame around the code
    xleftmargin=0pt,               % Remove left margin
    xrightmargin=0pt,              % Remove right margin
    aboveskip=5pt,                 % Reduce space above the code block
    belowskip=5pt                  % Reduce space below the code block
}

%---------------------------------------------------------
% Math, units, spacing, etc.
%---------------------------------------------------------
\usepackage{siunitx}
\usepackage{setspace}
\usepackage{textgreek}

% Add float package for "H" float option
\usepackage{float}

%---------------------------------------------------------
% Other packages
%---------------------------------------------------------
\usepackage{ifthen}
\usepackage{acronym}
\PassOptionsToPackage{hyphens}{url} % URLs in Hyperlinks umbrechen
\usepackage[breaklinks=true]{hyperref} 
\usepackage{array}                % Bessere Tabellenformatierung
\usepackage{enumitem}             % Kontrolle über Listen-Layouts
\usepackage{nomencl}
\usepackage{scrlayer-scrpage}     % Header und Footer

% Adjust header and footer heights
\setlength{\headheight}{14.5pt}
\setlength{\footheight}{34.16666pt}

%---------------------------------------------------------
% Bibliography (biblatex mit Biber)
%---------------------------------------------------------
\usepackage[backend=biber, style=numeric]{biblatex}  

\addbibresource{literatur.bib}  

%---------------------------------------------------------
% Platzhalter
%---------------------------------------------------------
\newcommand{\titel}{Umbau und Inbetriebnahme eines Konzeptfahrzeugs zur Erprobung eines neuen Fahrantriebs}
\newcommand{\untertitel}{}
\newcommand{\arbeit}{T3000 Hausarbeit}
\newcommand{\studiengang}{Elektrotechnik}
\newcommand{\studienrichtung}{Fahrzeugelektronik}
\newcommand{\autor}{Luka Tadic}
\newcommand{\abgabe}{14.04.2025}
\newcommand{\bearbeitungszeitraum}{19.01.2025 - 14.04.2025}
\newcommand{\matrikelnr}{5726700}
\newcommand{\kurs}{TFE22-1}
\newcommand{\firma}{Kramer Werke GmbH}
\newcommand{\betreuerfirma}{Dipl. Ing. Christian Borgmann}
\newcommand{\gutachterdhbw}{Prof. Dr. Ing. Konrad Reif}
\newcommand{\jahr}{2025}

%---------------------------------------------------------
% Header und Footer mit Linien
%---------------------------------------------------------
\clearpairofpagestyles         % Standard-Stile löschen

% Header with consistent logo placement for two logos and line position
\ohead{%
    \parbox{\textwidth}{% Create a flexible container for both images
        \raisebox{1.5cm}[0pt][0pt]{% Raise the Kramer logo
            \includegraphics[width=3cm]{images/kramer.png}%
        }%
        \hfill % Horizontal space between the two logos
        \raisebox{1.5cm}[0pt][0pt]{% Raise the DHBW logo
            \includegraphics[width=3cm]{images/DHBW_d_R_FN_46mm_4c}%
        }%
    }%
    \\[-1.5cm] % Move the header line down
    \rule{\textwidth}{0.4pt} % Horizontal rule for the header line
}


% Footer with consistent alignment and contents below the line
\setkomafont{pagefoot}{\normalfont} % Ensure consistent font style
\cfoot{%
    \rule{\textwidth}{0.4pt}\\ % Horizontal rule
    \vspace{0.3em} % Small vertical space
    \begin{tabular}{@{}p{0.33\textwidth}p{0.33\textwidth}p{0.33\textwidth}@{}}
        \arbeit & \centering \autor & \raggedleft \thepage
    \end{tabular}
}


\pagestyle{scrheadings}        % Stil aktivieren

%---------------------------------------------------------
% Dokumentbeginn
%---------------------------------------------------------
\begin{document}
\sloppy

%---------------------------------------------------------
% Titelseite
%---------------------------------------------------------
\thispagestyle{empty}  % Kein Header oder Footer auf der Titelseite
\hypersetup{pageanchor=false}

\begin{titlepage}
\enlargethispage{4.0cm}
\sffamily  % Serifenlose Schrift für die Titelseite

% Create a container for both logos at the top
\parbox{0.5\linewidth}{%
    \begin{flushleft}
        \includegraphics[width=0.4\linewidth]{images/kramer.png}\\[5ex] % Kramer logo on the left
    \end{flushleft}
}
\parbox{0.5\linewidth}{%
    \begin{flushright}
        \includegraphics[width=0.4\linewidth]{images/DHBW_d_R_FN_46mm_4c}\\[5ex] % DHBW logo on the right
    \end{flushright}
}

% Title and information in the center
\begin{center}

{\fontsize{20.74pt}{24pt}\selectfont
\textbf{\titel}\\[1.5ex]}

{\fontsize{17pt}{20pt}\selectfont
\textbf{\arbeit}\\[2ex]}

{\fontsize{14pt}{17pt}\selectfont
Studiengang \studiengang\\[2ex]}

{\fontsize{12pt}{14pt}\selectfont
Studienrichtung \studienrichtung\\[1ex]
Duale Hochschule Baden-Württemberg Ravensburg, Campus Friedrichshafen\\[5ex]
von\\[1ex]
\autor\\[15ex]}

\end{center}

% Footer-like table with additional information
\begin{center}
{\fontsize{12pt}{14pt}\selectfont
\begin{tabular}{ll}
Abgabedatum:                    & \quad \abgabe \\  
Bearbeitungszeitraum:           & \quad \bearbeitungszeitraum \\  
Matrikelnummer:                 & \quad \matrikelnr \\  
Kurs:                           & \quad \kurs \\  
Dualer Partner:                 & \quad \firma \\ % entfällt bei Studienarbeit
Betreuerin / Betreuer:          & \quad \betreuerfirma \\  
Gutachterin / Gutachter:        & \quad \gutachterdhbw \\ [2ex]
\end{tabular}
}
\end{center}

\end{titlepage}

\clearpage

\pagestyle{scrheadings}  % Header und Footer nach Titelseite aktivieren
\hypersetup{pageanchor=true}

%---------------------------------------------------------
% Erklärung
%---------------------------------------------------------

\pagenumbering{Roman}

\section*{Erklärung}

Ich versichere hiermit, dass ich meine \arbeit\ mit dem Thema:

\begin{quote}
    \textit{\titel}
\end{quote}

selbstständig verfasst und keine anderen als die angegebenen Quellen und Hilfsmittel benutzt habe.  
Ich versichere zudem, dass die eingereichte elektronische Fassung mit der gedruckten Fassung übereinstimmt.\\[6ex]

Friedrichshafen, den \today \\[1ex]
\rule[-0.2cm]{5cm}{0.5pt} \\  
\autor \\[10ex]

\rmfamily

\clearpage

\section*{Kurzfassung}
\begin{spacing}{1.8}  % Adjust line spacing
    \fontsize{14pt}{15pt}\selectfont  % Font size and line spacing
English translation of the “Kurzfassung”.

\end{spacing}

\clearpage
%---------------------------------------------------------
% Abstract
%---------------------------------------------------------
\section*{Abstract}
\begin{spacing}{1.8}  % Adjust line spacing
    \fontsize{14pt}{15pt}\selectfont  % Font size and line spacing
English translation of the “Kurzfassung”.

\end{spacing}
\clearpage

% List of figures
\listoffigures

\clearpage

\section*{Abbkürzungsverzeichnis}
\begin{spacing}{1.8}  % Adjust line spacing
    \fontsize{14pt}{15pt}\selectfont  % Font size and line spacing



\end{spacing}

\clearpage


%---------------------------------------------------------
% Inhaltsverzeichnis
%---------------------------------------------------------
\tableofcontents

\clearpage
\pagenumbering{arabic}


\section{Einleitung und Motivation}
Nutzung Best Point Software (Risikominimierung)

Einführung AMA-Keypads (Kostenreduzierung)

Einführung Servobremse

Danfoss BPC Fahrantrieb (Zielpreis weit unterschritten)

Mehrkosten kompensiert

Höheres Drehmoment

Stärkere Dieseldrehzahlabsenkung auf 1800 U/min (verringerte Geräusche + geringerer Kraftstoffverbrauch) (20 \% Roadmodus, 10 \% Andwendungsmix)

Unterschiedliche Faharmodi

ICVD Getriebeeinheit

Schon im W01/W02 vorhanden

Ansatzpunkt (Leistung, Dynamik, Reststeigfähigkeit)

verbesserte Reststeigfähigkeit der Maschine (Erhöhung der Leistung des Diesels)

Für die Umsetzung des Best Point Controls wird die Pumpe mit einem Schwenkwinkelsensor ausgeführt

ICVD best Point (geänderte Steuerdeckel, Übernahme W01/W02, Kosteneffizienz, mehr Dieseldrehzahlabsenkung)

neuer Servo Bremskonzept

Hill hold funktion


\clearpage

\section{Zielsetzung}
\section{Ablauf Umbau und Inbetriebnahme}
\section{Ausblick}
\clearpage

%---------------------------------------------------------
% Bibliografie
%---------------------------------------------------------
\begingroup
\renewcommand{\bibfont}{\fontsize{13pt}{12pt}\selectfont}  
\sloppy
\nocite{*}
\printbibliography

\end{document}
